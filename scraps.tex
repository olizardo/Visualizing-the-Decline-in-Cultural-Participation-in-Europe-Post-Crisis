, we see that in Portugal (panels c-d), the increase in non-participation is concentrated among people with about a high-school education (16–19 years old when they completed schooling), who went from a non-participation rate of 22\% in 2007 to 40\% in 2013. Respondents with a university education were relatively protected (14\% versus 16\%); in fact, for this group, the proportion engaging in the most activities (five to six) seems to have grown post-crisis. In Spain (panels e-f), we find the same pattern, although with a less steep increase in non-participation going from 17\% to 25\% among the high-school educated. Italy tells the same story, although there we also observe relatively big increases in non-participation for those with less than a high-school education, who go from 47\% with zero cultural activities in 2007 to 58\% in 2013. 

